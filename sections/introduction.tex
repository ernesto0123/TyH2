\documentclass[crop=false]{standalone}
\usepackage{graphicx}
\graphicspath{{images/}}
\usepackage[utf8]{inputenc}
\usepackage{blindtext}

\begin{document}


\begin{figure}
\centering
\includegraphics[width=3cm]{images/Escudo_UNCuyo.png}
\label{fig:lionfigure}
\caption{Escudo Oficial propuesto por el Rectorado de la UNCuyo}
\end{figure}



\section{Introducción}
En un escenario global donde la industria olivícola enfrenta crecientes demandas tanto en diversidad de productos como en sostenibilidad y eficiencia, emerge la necesidad de reinventar y modernizar las prácticas tradicionales. La ingeniería industrial, caracterizada por su enfoque sistémico y su capacidad para optimizar procesos, juega un papel esencial en la respuesta a estos desafíos.
El contexto macroeconómico y social de Argentina ha planteado retos significativos para el sector agropecuario. En particular, en Mendoza, estos desafíos han llevado a propietarios de olivares a parcelar sus tierras y transformarlas en barrios o, en otros casos, a cambiar completamente de rubro debido a la falta de rentabilidad que enfrenta la industria olivícola tradicional.
Ante esta realidad, el presente proyecto busca llevar a cabo una reingeniería integral de la plantación de olivos y del proceso productivo en una poliproductora de derivados del olivo. Esta reingeniería contempla la implementación de las técnicas más modernas de recolección, que conlleva una renovación profunda a través de la tala y replantación de olivos, manteniéndolos en un formato arbustivo para maximizar la eficiencia de la cosecha.
Nuestros objetivos se centran en desarrollar productos novedosos, de alto valor y altamente competitivos en el mercado internacional. Para lograrlo, se integrarán conceptos de economía circular, una gestión avanzada del recurso hídrico, el uso de fertilizantes naturales y una administración detallada de la plantación mediante tecnologías de punta, tales como la agricultura de precisión con drones, sensores y sistemas innovadores de riego.
Con Mendoza como punto focal, este proyecto aspira no solo a revitalizar la industria olivícola en la región sino también a establecer un modelo productivo sostenible, responsable y alineado con las demandas y expectativas del consumidor del siglo XXI.

\end{document}

